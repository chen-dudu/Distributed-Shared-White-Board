\documentclass{article}
% \documentstyle[fullpage]{article}
\usepackage[utf8]{inputenc}
\usepackage{amsfonts}
\usepackage{amssymb}
\usepackage{geometry}
\usepackage{graphicx}

\geometry{a4paper, top = 25mm, bottom = 50mm, total = {6in, 8in}}

\renewcommand{\theenumi}{\alph{enumi}}
    
\title{COMP90015 Distributed Systems \\ 
       Assignment 2 Report}
\author{Liguo Chen\\Student ID: 851090}
\date{\today}

\begin{document}

\maketitle

\section*{Problem}
In this project, a shared whiteboard will be built, which supports multiple users drawing simultaneously.\\\\
Some basic features include:
\begin{itemize}
    \item draw line, oval, circle, and rectangle
    \item insert texts
    \item users can choose the favourite colors for drawing and inserted text
\end{itemize}
Some advanced features include:
\begin{itemize}
    \item users can chat with each other
    \item manager can kick out ordinary users
    \item users can leave
    \item manager can save the whiteboard on local machine
\end{itemize}

\section*{System architecture}
The system is Client - Server architecture.\\\\
All information about whiteboard, including active users, chat messages, content of the canvas, is stored in the central server.\\\\
Information exchange, for example manager kicks out a user, is done via the central server.

\section*{Communication}
The Client - Server communication is done by using java RMI.\\\\
\textbf{iRemote} is the remote interface, which defines available methods the server prodvides. \textbf{Remoteo} is the remote object, which implements all the remote methdos defined in \textbf{iRemote}, and is the main server-side end-point for communication.

\section*{Design diagrams}
\subsection*{Class diagram}
\subsubsection*{Overall}
\includegraphics[width=16cm]{overall}
\subsubsection*{MyObj}
\includegraphics[width=8cm]{myobj}
\subsubsection*{Client}
\includegraphics[width=8cm]{client}
\subsubsection*{iRemote}
\includegraphics[width=8cm]{iremote}
\subsubsection*{Remoteo}
\includegraphics[width=8cm]{remoteo}
\subsubsection*{Windows}
\includegraphics[width=8cm]{windows}
\subsubsection*{Canvas}
\includegraphics[width=8cm]{canvas}
\subsubsection*{UserList}
\includegraphics[width=8cm]{userlist}
\subsubsection*{ChatBox}
\includegraphics[width=8cm]{chatbox}

\subsection*{Sequence Diagram}
\includegraphics[width=16cm]{sequence diagram}

\section*{Implementation details}
\textbf{MyObj} represents an object in the canvas. It could be a shape, or an inserted text. To support java RMI communication, \textbf{MyObj} is serializable.\\\\
\textbf{Client} represents a user in the system. It is responsible for notifying server about modifications and getting the latest information from server and passing down to window for rendering.\\\\
\textbf{Windows} represents the client GUI window. It communicate with \textbf{Client} to fulfill all the required features of the system.\\\\
\textbf{Canvas} represents the canvas object in the client GUI. When a new shape is drawn or a new text is inserted, it will notify \textbf{Windows}\\\\
\textbf{UserList} represents a component in the GUI and its main job is to display a list of active users in the system\\\\
\textbf{ChatBox} is a GUI component similar to \textbf{UserList}. Instead of displaying a list of users, \textbf{ChatBox} shows a list of chat message sent by all users.

\end{document}
